
\documentclass[12pt]{article}
\usepackage{fullpage,enumitem,amsmath,amssymb,graphicx,bm,listings,algpseudocode,hyperref,url}
\newcommand{\vect}[1]{\boldsymbol{#1}}

\lstdefinestyle{custom}{
  basicstyle=\footnotesize\ttfamily,
  language=Python,
}
\begin{document}

\title{Stock forecast system with Elliott Wave pattern recognition and adaptive trading strategy}
\author{Motonari ITO}

\maketitle
\begin{center}
\begin{tabular}{rl}
SUNet ID: & motonari \\
Collaborators: & Sundararaman Shiva
\end{tabular}
\end{center}

\begin{abstract}
  We build a system to advise the best trading strategy. The system
  consists of two components. One is stock price predictors based on
  reflex models. The other is a trading strategy optimizer based on a
  state model. We also build a test tool to evaluate the performance
  of each component.
\end{abstract}

\section{Introduction}

Stock forecast has been studied and practiced with various degree of
success. The technical analysis is a methodology based on the
historical stock market prices \cite{wiki:technical_analysis}. The
fundamental analysis tries to predict based on the business's
financial statement \cite{wiki:fundamental_analysis}. Data mining over
the Internet with sentiment analysis also became popular recently
\cite{web:data_mining_analysis}.

Elliott Wave Principle (EWP) is a classical technical analysis method
\cite{frost1981elliott, web:study_of_cycles}. It is a hypothesis that stock market price
can be modeled as a sequence of waves which shapes follow some defined
rules. EWP suggests we can predict the future market price more
accurately than a random chance by recognizing the wave pattern.  This
is distinct from other stock price prediction methods in that it
relies solely on the historical price changes and doesn't use external
information such as market sentiment or industrial news. 

EWP, has been criticized for the poor performance
\cite{aronson2007evidence-based}. Notably, for a given stock
historical data, the rules yields many different interpretation of the
wave forms. This uncertainty makes the future prediction hard, if not
impossible, while one can claim the accuracy of the theory
\emph{after} the fact. In a sense, EWP is so powerful and complex
model that one cannot avoid over-fitting.

We, however, believe the essence of EWP is still valid; future price
is influenced by past price pattern. Intuitively, some often predict
the price to go up if the past price has strong upward trend. Others
may predict downward trend if the past price shows inverted-V
shape. While such prediction may not be an inherent property of the
stock market, the fact that many people believe that way affect the
market. Therefore, it is probable that a reflex based machine learning
algorithm can predict a future price based on the past prices.

TODO: talk about sentiment analysis based approach

Given that we have a sensible predictor, it is still an open question
when and how to trade stocks to optimize the asset because the
predictor is inherently imperfect. For example, it may not be smart to
sell the entire stocks immediately just because a predictor says price
would go down, because the prediction could be wrong. The trading
decision should be educated by the actual performance of predictors.

Our approach is to use a state based learning algorithm to find the
optimal trading strategy. Intuitively, as it runs the predictors on
the historical data, the trader will learn the peculiarity of each
predictor. 


\section{System}

\subsection{Overview}

The system consists of two parts: predictors and traders. A predictor
forecasts future prices of a particular stock such as AAPL. A trader
learns the optimal trading strategy (i.e. when to buy/sell how much
stocks) based on predictors and the actual stock prices.

There are three predictors.

\begin{description}
  \item[SimpleNNPredictor] It uses a multilayer perceptron
    algorithm. The input is the prior stock prices and the output is a
    set of future prices.
  \item[LinearPredictor] It uses a linear regression with stochastic
    gradient descent algorithm. The input is the prior stock prices
    and the output is a set of future prices.
  \item[SentimentPredictor] It uses New York Times Community API
    \cite{web:nytimes_community_api}to retrieve the customer comments
    on each news article. Then, it uses Stanford Core NLP
    \cite{manning-EtAl:2014:P14-5} to find the sentiment of each
    comments.
\end{description}

In addition, we have \verb|CheatPredictor| which returns the future
prices based on the actual data. It's used for evaluating the
performance of trader components.

There are two traders.

\begin{description}
  \item[RoteQTrader] It uses Q-Learning algorithm without function
    approximation; uses simplified state definition instead. 

  \item [QTrader] It uses Q-Learning algorithm with a function
    approximation.
\end{description}

A trader receives predictors output and use them as a state. 

\subsection{Predictors}



\subsection{Trader}


If we had a perfect predictor, the optimal strategy is to buy before
the stock price goes up and to sell before it goes down. However, no
predictor is perfect.

Our goal is to maximize our asset after the sequence of
trades. Suppose the last day index is $n$, this is defined as follows.

\[
AssetValue = p_n(O_n + C_n)
\]
, where $p_n$ is the stock price of the last day, $O_n$ is the number
of stocks owned at the last day, and $C_n$ is the maximum number of
stocks we could buy with our current cash in the pocket at the last
day.

The trader finds the optimal strategy to meet the goal.

\subsubsection{Model}

We model it as MDP where we don't know the transition function.

\begin{description}
\item[State] The state consists of these numbers. 
  \begin{itemize}
  \item $O_k$, the number of stocks owned.
  \item $C_k$, the number of stocks (floating number) we can buy with
    our current cash amount. In other words, it is $CashAmount/P_{k}$
    where $P_k$ is the current stock price.
    \item $m_k$, the predicted slope by performing a least square
      polynomial fit over the predicted future price changes.
    \item $residual_k$, the sum of residuals of the predicted slope
      above.
  \end{itemize}

\item[Initial State] Before running MDP, we initialize the state as follows.
  \begin{itemize}
  \item $O_0 = 0$
  \item $C_0 = 10$
  \item $m_0, residuals_0$ are initialized by the predictor based on the first day.
  \end{itemize}
  
\item[Action] The action is an integer in the range: $[-O_k,C_k]$. The
  negative value means to sell owned stocks for that amount, and the
  positive value means to buy stocks for that amount.

\item[Transition] On taking a action, the state moves to the next
  day. Note that the system doesn't know the next state beyond today.

\item[Reward] The reward is the difference of asset value before and
  after the state transition. 
  
  \[
  Reward_k = P_k(O_{k} + C_{k}) - P_{k-1}(O_{k-1} + C_{k-1})
  \]
  
  Intuitively, we want to have more stocks when the stock price is
  high and we want to have more cash when the stock price is low.
\end{description}

\subsubsection{Algorithm}

We use Q-learning with a function approximation and epsilon-greedy
learning.


\section{Performance}

\subsection{Predictors}
\subsection{Trader}

\section{Comparison to Baseline and Oracle}

\section{Conclusion}


\bibliographystyle{plain}
\bibliography{final,StanfordCoreNlp2014}

\end{document}



