\documentclass[twocolumn,10pt]{asme2ej}
% \usepackage{fullpage,enumitem,amsmath,amssymb,graphicx,bm,listings,algpseudocode,hyperref,url,multirow,caption}
\usepackage{enumitem,amsmath,amssymb,graphicx,bm,listings,algpseudocode,hyperref,url,multirow,caption}
\newcommand{\vect}[1]{\boldsymbol{#1}}

\lstdefinestyle{custom}{
  basicstyle=\footnotesize\ttfamily,
  language=Python,
}
\begin{document}

\title{Stock forecast system with Elliott Wave pattern recognition and adaptive trading strategy}
\author{Motonari ITO
  \affiliation{
    SUNet ID: motonari
    }
}

\maketitle

\begin{abstract}
  We build a system to advise the best trading strategy. The system
  consists of two components. One is stock price predictors based on
  reflex models. The other is a trading strategy optimizer based on a
  state model. We also build a test tool to evaluate the performance
  of each component.
\end{abstract}

\section{Introduction}

Stock forecast has been studied and practiced with various degree of
success. The technical analysis is a methodology based on the
historical stock market prices \cite{wiki:technical_analysis}. The
fundamental analysis tries to predict based on the business's
financial statement \cite{wiki:fundamental_analysis}. Data mining over
the Internet with sentiment analysis also became popular recently
\cite{web:data_mining_analysis}.

Elliott Wave Principle (EWP) is a classical technical analysis method
\cite{frost1981elliott, web:study_of_cycles}. It is a hypothesis that stock market price
can be modeled as a sequence of waves which shapes follow some defined
rules. EWP suggests we can predict the future market price more
accurately than a random chance by recognizing the wave pattern.  This
is distinct from other stock price prediction methods in that it
relies solely on the historical price changes and doesn't use external
information such as market sentiment or industrial news. 

EWP, has been criticized for the poor performance
\cite{aronson2007evidence-based}. Notably, for a given stock
historical data, the rules yields many different interpretation of the
wave forms. This uncertainty makes the future prediction hard, if not
impossible, while one can claim the accuracy of the theory
\emph{after} the fact. In a sense, EWP is so powerful and complex
model that one cannot avoid over-fitting.

We, however, believe the essence of EWP is still valid; future price
is influenced by past price pattern. Intuitively, some often predict
the price to go up if the past price has strong upward trend. Others
may predict downward trend if the past price shows inverted-V
shape. While such prediction may not be an inherent property of the
stock market, the fact that many people believe that way affect the
market. Therefore, it is probable that a reflex based machine learning
algorithm can predict a future price based on the past prices.

TODO: talk about sentiment analysis based approach

Given that we have a sensible predictor, it is still an open question
when and how to trade stocks to optimize the asset because the
predictor is inherently imperfect. For example, it may not be smart to
sell the entire stocks immediately just because a predictor says price
would go down, because the prediction could be wrong. The trading
decision should be educated by the actual performance of predictors.

Our approach is to use a state based learning algorithm to find the
optimal trading strategy. Intuitively, as it runs the predictors on
the historical data, the trader will learn the peculiarity of each
predictor. 

\section{System}

\subsection{Overview}

The system consists of two parts: predictors and traders.

A predictor forecasts a future price of a particular stock based on
various input data such as prior stock prices.

A trader uses the prediction and learns the optimal trading strategy
(i.e. when to buy/sell how much stocks).

There are four predictors (SimpleNNPredictor, LinearPredictor,
SentimentPredictor, and CheatPredictor) and one trader
(QTrader). The trader uses weighted average of predictions.

\subsection{Predictors}

All the predictors take a hyper parameter $D$, which requests the
predictor to predict the stock price change after $D$ days from the date.

\subsubsection{SimpleNNPredictor}
The predictor uses the multilayer perception implementation from
scikit-learn \cite{web:scikit_learn}. It extract the feature by
looking back the prior stock price.

The look back date is defined as an array variable $b$:

\[
b \gets [87, 54, 33, 21, 13, 8, 5, 3, 2, 1]
\]

In the training phase, it looks back the stock prices and calculate
the price changes $X$ compared to the current price, which is used as
an input to the algorithm.

\[
X_i \gets \left\{\frac{p_i - p_{i - b_j}}{p_{i - b_j}} : j \in |b|\right\}
\]

The target value is is the actual stock price change for the given
future date, \verb|np.datetime64(today)+D|.

\subsubsection{LinearPredictor}
The predictor is same as SimpleNNPredictor except that the underlying
algorithm uses a linear regression with stochastic gradient descent,
also from scikit-learn \cite{web:scikit_learn}.

\subsubsection{SentimentPredictor}

The predictor uses New York Times Community API
\cite{web:nytimes_community_api} to retrieve the customer comments of
each news article from Jan 1, 2010 to Nov 10, 2016.

Then, it uses Stanford Core NLP \cite{manning-EtAl:2014:P14-5} to find
the sentiment. The algorithm returns a tuple of sentimentValue and
sentiment for each sentence in the comment. For example, a very
positive sentence may return (3, 'positive'). A weakly negative
sentence would return (1, 'negative').

We estimate the sentiment of the comment by averaging the sentiment
values.

\begin{verbatim}
def comment_sentiment(comment):
  total = 0
  score = 0
  for each sentence in comment:
    if sentiment == 'positive':
      score += sentimentValue
      total += sentimentValue
    elif sentiment == 'negative':
      score -= sentimentValue
      total += sentimentValue
    else:
      total += sentimentValue

  return float(score) / total
\end{verbatim}

We take a comment which contains a related word to a stock symbol. For
example, for \verb|aapl|, we pick a comment with words 'aapl',
'apple', 'iphone', 'ipad', 'mac', 'ipod', or 'ios'.

Then, for a given date, we look back the last 30 days of the sentiment
for the stock and create a feature vector of size 30.

We use a linear regression with stochastic gradient descent from
scikit-learn \cite{web:scikit_learn} to train the predictor.The target
value is is the actual stock price change for the given future date,
\verb|np.datetime64(today)+D|.

\subsection{Trader}

If we had a perfect predictor, the optimal strategy is of course to
buy before the stock price goes up and to sell before it goes
down. However, no predictor is perfect.

Our goal is to maximize our asset value $U$ after the sequence of
trades. Suppose the last day index is $n$, the asset value $U$ is defined:

\[
U = p_n(o_n + c_n),
\]
where $p_n$ is the stock price of the last day, $o_n$ is the number
of stocks owned at the last day, and $c_n$ is the maximum number of
stocks we could buy with our current cash in the pocket at the last
day.

The trader finds the optimal strategy to meet the goal.

\subsubsection{Model}

We model it as MDP where we don't know the transition function.

\begin{description}
\item[State] For $i$-th day, the state is defined to have a tuple of
  these values.
  \begin{itemize}
  \item $o_i$, the number of stocks owned.
  \item $c_i$, the number of stocks (floating number) we can buy with
    our current cash amount. In other words, it is the cash amount
    divided by the current stock price $p_i$.
    \item $m_i$, the predicted slope by performing a least square
      polynomial fit over the predicted future price changes.
    \item $r_i$, the sum of residuals of the predicted slope
      above.
  \end{itemize}

\item[Initial State] Before running MDP starting at day index $start$,
  we initialize the state as follows.
  \begin{itemize}
  \item $o_{start} = 0$
  \item $o_{start} = 10$
  \item $m_{start}$ and $r_{start}$ are initialized by the predictor based
    on the first day.
  \end{itemize}
  
\item[Action] The action is an integer in the range: $[-o_i,c_i]$. The
  negative value means to sell owned stocks for that amount, and the
  positive value means to buy stocks for that amount.

\item[Transition] On taking a action, the state moves to the next
  day. Note that the system doesn't know the next state beyond today.

\item[Reward] The reward is the difference of asset value before and
  after the state transition. 
  
  \[
  Reward = p_i(o_{i} + c_{i}) - p_{i-1}(o_{i-1} + c_{i-1})
  \]
  
  Intuitively, we want to have more stocks when the stock price is
  high and we want to have more cash when the stock price is low.
\end{description}

\subsubsection{Algorithm}

We use Q-learning with a function approximation and epsilon-greedy
learning.

\subsubsection*{Parameters}

\begin{align*}
  \epsilon &:= \text{Parameter for epsilon-greedy policy} \\
  \phi(state,action) &:= \text{feature extractor} \\
  p_i &:= \text{stock price of i-th day} \\
  \vect{w} &:= \text{weight to learn} \\
  \hat{Q}_{opt}(s, a; \vect{w}) &:= \vect{w} \dot \phi(s, a) \\
\end{align*}

In learning phase, for 100 times, we pick a day index randomly and run
the MDP for the next 90 days.

\subsubsection*{Choose an action}

We examine the state and obtain the available actions: $[-o_k,c_k]$.

Based on whether a random number $[0.0, 1.0]$ is greater than
$\epsilon$, we pick exploration or exploitation.

\[
  \pi_{act}(s) = \\
  \begin{cases}
    \arg \max_{a \in actions}\hat{Q}_{opt}(s, a) & \text{probability } 1-\epsilon \\
    \text{random choice from actions} & \text{probability } \epsilon \\
  \end{cases}
\]


\subsubsection*{Calculate reward}

Based on the action, we calculate the reward.

\[
  reward = p_{i+1}(o_{i+1} + c_{i+1}) - p_{i}(o_{i} + c_{i})
\]

\subsubsection*{Transition to the next state}

When an action is taken, MDP moves to the next state, which represents
the stock market and the current asset of the next day.

\begin{align*}
  c_{i+1} &\gets (c_{i} - action)\frac{p_i}{p_{i+1}} \\
  o_{i+1} &\gets o_{i} + action \\
  m_{i+1}, r_{i+1} &\gets \text{predict(i+1)} \\
\end{align*}

\subsubsection*{Update weights}

We update the weights as follows.
\begin{align*}
\hat{V}_{opt}(s') &\gets \max_{a \in actions(s')}\hat{Q}_{opt}(s', a;\vect{w}) \\
\vect{w} &\gets \vect{w} - \eta[\hat{Q}_{opt}(s, a;\vect{w}) - (reward + \gamma \hat{V}_{opt}(s'))\phi(s,a)
\end{align*}

\subsubsection{Test}

In test phase, we run the learning algorithm on a new data with the
following modification.

\begin{itemize}
\item Always choose the optimal action, that is to set $\epsilon = 0$
\item Skip weight update step.
\end{itemize}

Then, see if how much money earned or lost by looking at the final
asset value.


\section{Performance}

\subsection{Predictors}

We use stock price data from Yahoo Finance
\cite{web:yahoo_finance}. It provides day-to-day closing stock price
for the period shown in table \ref{yahooStockData}.

\begin{table}
  \begin{tabular}{ccc}
    Symbol & Start Date & End Date \\
    \hline
    aapl & 1980-12-12 & 2016-11-11 \\
    bp & 1977-01-03 & 2016-11-11 \\
    cop & 1981-12-31 & 2016-11-11 \\
    cost & 1986-07-09 & 2016-11-11 \\
    cvx & 1970-01-02 & 2016-11-11 \\
    dj & 1985-01-29 & 2016-11-11 \\
    gdx & 2006-10-02 & 2008-03-10 \\
    hd & 2011-11-14 & 2016-11-11 \\
    ibm & 1962-01-02 & 2016-11-11 \\
    ko & 1962-01-02 & 2016-11-11 \\
    low & 2011-11-14 & 2016-11-11 \\
    nke & 1980-12-02 & 2016-11-11 \\
    qcom & 1991-12-13 & 2016-11-14 \\
    rut & 2011-10-03 & 2016-02-08 \\
    tgt & 1980-03-17 & 2016-11-11 \\
    wmt & 2011-11-14 & 2016-11-11 \\
    xcom & 1970-01-02 & 2016-11-11 \\
  \end{tabular}
  \caption{Yahoo Finance Stock Data}
  \label{yahooStockData}
\end{table}



We measure the performance of predictors by whether it predicts
upward/downward trend correctly on the test data. For example, if the
predictor predicts upward trend correctly, we call it as true
positive.

Table \ref{pperf630} shows the predictor performance after 630
iteration over the data set.



\begin{table}
  \begin{tabular}{c|cccccc}
    & TP & FN & FP & TN & Accuracy & F1 Score \\
    \hline

    SimpleNNPredictor & 0.43 & 0.08 & 0.42 & 0.07 & 0.50 & 0.63 \\
    LinearPredictor & 0.51 & 0.00 & 0.49 & 0.00 & 0.51 & 0.68 \\

  \end{tabular}
  \caption{Predictor performance after 630 iteration}
  \label{pperf630}
\end{table}


\begin{table}
  \begin{tabular}{cc}
    \begin{tabular}{cc|cc}
      & & \multicolumn{2}{c}{Predicted} \\
      & & $+ $ & $-$ \\
      \hline
      \multirow{2}{*}{Actual}
      & $+$ & 0.20 & 0.15 \\
      & $-$ & 0.39 & 0.26 \\
      \hline
    \end{tabular}
    &
    \begin{tabular}{cc}
      Accuracy & 0.46 \\
      F1 Score & 0.43 \\
    \end{tabular}
  \end{tabular}
  \caption{instance with 1 days delta}
\end{table}


\subsubsection{SimpleNNPredictor}


\subsubsection{LinearPredictor}

\subsection{Trader}

We measure the performance of the trader by plugging the perfect
predictor: \verb|CheatPredictor|. 

\section{Comparison to Baseline and Oracle}

\section{Conclusion}


\bibliographystyle{plain}
\bibliography{final,StanfordCoreNlp2014}

\end{document}



